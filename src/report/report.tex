\documentclass[a4paper,12pt]{article}
\usepackage{times}
\usepackage{hyperref}
\usepackage{xcolor}

\usepackage{geometry}
\geometry{
    a4paper,
    margin=1in,
 }

\usepackage{listings}
\usepackage{xcolor}

\definecolor{codegreen}{rgb}{0,0.6,0}
\definecolor{codegray}{rgb}{0.5,0.5,0.5}
\definecolor{codepurple}{rgb}{0.58,0,0.82}
\definecolor{backcolour}{rgb}{0.95,0.95,0.92}

\lstdefinestyle{cppstyle}{
    language=C++,
    backgroundcolor=\color{backcolour},   
    commentstyle=\color{codegreen},
    keywordstyle=\color{magenta},
    numberstyle=\tiny\color{codegray},
    stringstyle=\color{codepurple},
    basicstyle=\ttfamily\footnotesize,
    breakatwhitespace=false,         
    breaklines=true,                 
    captionpos=b,                    
    keepspaces=true,                 
    numbers=left,                    
    numbersep=5pt,                  
    showspaces=false,                
    showstringspaces=false,
    showtabs=false,                  
    tabsize=2
}

\lstdefinestyle{pystyle}{
    language=Python,
    backgroundcolor=\color{backcolour},   
    commentstyle=\color{codegreen},
    keywordstyle=\color{magenta},
    numberstyle=\tiny\color{codegray},
    stringstyle=\color{codepurple},
    basicstyle=\ttfamily\footnotesize,
    breakatwhitespace=false,         
    breaklines=true,                 
    captionpos=b,                    
    keepspaces=true,                 
    numbers=left,                    
    numbersep=5pt,                  
    showspaces=false,                
    showstringspaces=false,
    showtabs=false,                  
    tabsize=2
}

\lstset{style=cppstyle}

\hypersetup{
    colorlinks,
    citecolor=black,
    linkcolor=black,
    urlcolor=black
}

\begin{document}

\begin{titlepage}
    \begin{center}
        \vspace*{1cm}

        \LARGE
        \textbf{Take Home Lab 1}

        \vspace{0.5cm}

        \large
        CS4532 - Concurrent Programming

        \vspace{1.5cm}

        \textbf{190257C - Thamindu Sathruwan\\190332D - Sasitha Kumarasinghe}

        \vfill

        \normalsize
        Department of Computer Science and Engineering\\
        University of Moratuwa\\
        \today

    \end{center}
\end{titlepage}

\tableofcontents
\newpage

\section{Introduction}

This report is about the implementation of concurrent linked lists using mutexes and read-write locks from the \lstinline|pthreads| library. The report will discuss the implementation of linked lists, the configuration of the experiments, the results of the experiments, and a discussion of the results.

\section{Implementation}

\section{Implementation}

The implementation of the linked lists is done in C++ and the code is available in the \lstinline|src| directory. The code is organized into the following files:

\begin{itemize}
    \item \textbf{\lstinline|linked_list.h|:} Contains the class interfaces for the linked list implementations.
    \item \textbf{\lstinline|linked_list.cpp|:} Contains the implementation of the linked list classes.
\end{itemize}

\subsection{Linked List Interface}

Firstly, an abstract class, \lstinline|LinkedList|, is created as the interface for the linked list implementations. Following the sample code provided, a struct \lstinline|Node| is created as a protected member of the \lstinline|LinkedList| class to represent the nodes of the linked list.

\begin{lstlisting}
class LinkedList
{
public:
    LinkedList() : head(nullptr), length(0) {}

    virtual ~LinkedList()
    {
        Node *curr = head;
        Node *next = nullptr;
        while (curr != nullptr)
        {
            next = curr->next;
            delete curr;
            curr = next;
        }
    }

    void Populate(int n, int mInsert, int mDelete,
                  std::set<int> &insertVals, std::set<int> &deleteVals);

    virtual bool Member(int data) = 0;
    virtual bool Insert(int data) = 0;
    virtual bool Remove(int data) = 0;

protected:
    typedef struct Node
    {
        int data;
        Node *next;
    } Node;

    Node *head;
    int length;
};
\end{lstlisting}
\begin{center}
    \textbf{Listing 2.1.1:} \lstinline|LinkedList| class interface
\end{center}

\subsection{Populating the Linked List}

The \lstinline|LinkedList::Populate| method is implemented to populate the linked list with random values. The random seed is taken as the current time and the random number generator is seeded with it.

The linked list is populated as a sorted linked list. For this purpose, a \lstinline|std::set| is used to store the values that are inserted into the linked list. This is done to ensure that the linked list is populated with unique values. Since the \lstinline|std::set| is sorted, the values are inserted into the linked list in sorted order.

Even though the \lstinline|LinkedList::Insert| method is implemented to insert values in the ascending order, it is not used to populate the linked list. The reason is that this function takes a \( O(n) \) time in the worst case. Therefore, to insert \( n \) values, the time complexity would be \( O(n^2) \). The \lstinline|std::set| is implemented using a Red-Black tree which has a worst-case time complexity of \( O(\log n) \). Therefore, the time complexity of inserting \( n \) values into the \lstinline|std::set| is \( O(n \log n) \). This is much better than the \( O(n^2) \) time complexity of inserting values using the \lstinline|LinkedList::Insert| method.

Also, this method is used to create two lists containing random values to be inserted and deleted from the linked list while performing the operations. Again, the \lstinline|std::set| is used to ensure that the values are unique.

\section{Configuration}

\section{Results}

\section{Discussion}

\end{document}
